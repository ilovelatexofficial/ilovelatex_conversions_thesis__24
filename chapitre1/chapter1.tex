\titleformat
{\chapter}
[display]
{}
{ \rule{\textwidth}{2pt}
  \vspace{-2.7ex}
    \centering
\MakeUppercase{C\footnotesize{hapitre} \ \Large\thechapter}}
{0.05ex}
{
    \rule{\textwidth}{0.5pt}
    \vspace{1ex}
    \centering
    \bfseries\Large
}
[
\vspace{-0.5ex}%
\rule{\textwidth}{2pt}
]
\renewcommand \thesection {\arabic{chapter}.\arabic{section}}
\renewcommand \thechapter {\arabic{chapter}}
\fancyhead[LO]{ {\footnotesize\leftmark} }
\chapter{Présentation de l'entreprise GERMA GLACES}
\minitoc

\section{Introduction}

Ce premier chapitre a pour objectif de présenter l'entreprise GERMA GLACES, qui constitue le cadre organisationnel et opérationnel du projet de gestion du parc informatique. Il s'agit de situer le contexte dans lequel s'inscrit le système d'information à développer, en décrivant la nature de l'activité, la structure interne, les principaux services, ainsi que le rôle du service informatique au sein de l'entreprise.\par

Cette présentation est indispensable pour comprendre les besoins spécifiques de GERMA GLACES, les contraintes auxquelles elle est confrontée, ainsi que les enjeux liés à la modernisation de ses outils de gestion. Elle permet également de mieux justifier les choix techniques et fonctionnels qui seront détaillés dans les chapitres suivants.\par

\section{Fiche d'identité de l'entreprise}

GERMA GLACES est une entreprise algérienne opérant dans le secteur de l'agroalimentaire, plus précisément dans la fabrication et la distribution de produits glacés et desserts surgelés. Les principales informations d'identification de l'entreprise sont résumées dans le tableau ci-dessous.\par

L'entreprise dispose d'un \textbf{seul site principal} , regroupant les fonctions de direction, les bureaux administratifs, les unités de production, les espaces de stockage et les activités logistiques. Ce regroupement sur un site unique simplifie certains aspects de la communication interne, mais rend d'autant plus critique la disponibilité de l'infrastructure informatique utilisée par l'ensemble des services.\par

\section{Historique et mission de l'entreprise}

GERMA GLACES s'inscrit dans le secteur agroalimentaire avec une spécialisation dans la fabrication de glaces et de desserts surgelés. L'entreprise s'est progressivement développée pour répondre à une demande croissante en produits surgelés, notamment durant les saisons à forte consommation et dans le cadre d'une distribution régionale.\par

Même si ce mémoire n'a pas vocation à retracer en détail toute l'histoire de l'entreprise, il est important de souligner que sa croissance s'est accompagnée d'une \textbf{augmentation significative du volume d'activité} : demande plus importante, diversité accrue des produits, exigences renforcées en matière de qualité et de traçabilité.\par

La mission principale de GERMA GLACES peut être résumée comme suit :\par

\begin{itemize}
\item Offrir des produits alimentaires surgelés de qualité supérieure, conformes aux normes d'hygiène et de sécurité.
\item Assurer une distribution efficace dans la région de Boumerdès et, progressivement, dans d'autres zones géographiques.
\item Optimiser les processus internes par une gestion rigoureuse des ressources matérielles, humaines et logistiques.
\end{itemize}

Cette mission implique un fonctionnement coordonné entre les différents services, et, de plus en plus, un recours à des outils numériques pour planifier la production, suivre les stocks, gérer les commandes et piloter la performance.\par

\section{Secteur d'activité et gamme de produits}

GERMA GLACES opère dans le secteur de la production alimentaire, avec une gamme de produits variée répondant à différents segments de clientèle. Parmi les principales catégories de produits, on peut citer :\par

\begin{itemize}
\item Glaces en bac
\item Glaces individuelles
\item Cornets glacés
\item Desserts surgelés
\end{itemize}

La diversité de cette gamme implique :\par

\begin{itemize}
\item une organisation précise des recettes et des procédés de fabrication,
\item une gestion rigoureuse de la chaîne du froid,
\item un contrôle qualité constant,
\item une synchronisation entre production, stockage et distribution.
\end{itemize}

Le suivi des lots, des dates de péremption, des quantités produites et des retours clients nécessite un minimum d'outils informatisés (logiciels de gestion, tableurs, documents numériques), même dans une PME. C'est dans ce contexte que la gestion du parc informatique devient un enjeu important pour l'entreprise.\par

\section{Organisation interne de l'entreprise}

Afin d'assurer le bon déroulement de ses activités, GERMA GLACES est structurée en plusieurs \textbf{départements} , chacun ayant des missions spécifiques :\par

\begin{itemize}
\item \textbf{Direction Générale} : assure la vision stratégique, la prise de décision, la gestion globale de l'entreprise.
\item \textbf{Département Production} : gère l'ensemble du processus de fabrication des glaces et desserts surgelés.
\item \textbf{Département Qualité} : veille au respect des normes d'hygiène, de sécurité alimentaire et de conformité des produits.
\item \textbf{Département Logistique} : assure la gestion des stocks, des entrepôts et de la distribution.
\item \textbf{Département Commercial et Administratif} : prend en charge la gestion des commandes, la relation client, la facturation et les tâches administratives.
\item \textbf{Département Informatique et Support Technique} : responsable de l'infrastructure informatique, du support aux utilisateurs et de la sécurité des données.
\end{itemize}

Cette organisation permet de répartir clairement les responsabilités, tout en exigeant une \textbf{coordination étroite} entre les services. Par exemple, la production dépend des prévisions établies par le service commercial, qui lui-même s'appuie sur des données remontées du terrain.\par

\section{Processus opérationnels principaux}

Le fonctionnement quotidien de GERMA GLACES s'appuie sur plusieurs processus clés, parmi lesquels :\par

\begin{itemize}
\item \textbf{Processus de production} : planification, préparation des matières premières, fabrication, conditionnement, contrôle qualité.
\item \textbf{Processus logistique} : gestion des stocks, organisation des livraisons, suivi des flux physiques.
\item \textbf{Processus commercial} : prise de commandes, relations clients, facturation.
\item \textbf{Processus administratifs} : comptabilité, paie, gestion des ressources humaines.
\end{itemize}

Même si tous ces processus ne sont pas entièrement informatisés, ils reposent largement sur l'infrastructure informatique : postes utilisateurs, imprimantes réseau, outils bureautiques, logiciels de gestion.\par

\section{Présentation du service informatique}

Le \textbf{service informatique} joue un rôle central dans le fonctionnement de l'entreprise, bien qu'il soit réduit en effectif. Il est composé de :\par

\begin{itemize}
\item 1 Responsable informatique
\item 2 Techniciens support
\item 1 Administrateur réseau
\item 1 Développeur interne ou ressource polyvalente
\end{itemize}

Ses missions principales sont :\par

\begin{itemize}
\item garantir la disponibilité des équipements,
\item assurer la maintenance préventive et corrective,
\item installer et mettre à jour les logiciels,
\item gérer la sécurité informatique,
\item accompagner les utilisateurs au quotidien,
\item participer à l'évolution des outils numériques.
\end{itemize}

Cependant, en l'absence d'un système d'information dédié à la gestion du parc informatique, ces missions sont souvent réalisées de manière informelle et sans traçabili té.\par

\section{Conclusion du chapitre}

Ce chapitre a permis de présenter l'entreprise GERMA GLACES, son secteur d'activité, son organisation interne et le rôle du service informatique. Cette contextualisation constitue la base du travail d'analyse présenté dans le chapitre suivant.\par