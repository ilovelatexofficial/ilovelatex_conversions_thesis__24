\titleformat
{\chapter}
[display]
{}
{ \rule{\textwidth}{4pt}
  \vspace{-1.7ex}
    \centering
\MakeUppercase{{{\Huge C}{\LARGE onclusion}}}}
{0.05ex}
{
    \vspace{1ex}
    \centering
    \color{white}
}
[
  \color{black}
\vspace{-0.5ex}%
\rule{\textwidth}{4pt}
]
\renewcommand{\thechapter}{}
\renewcommand \thesection {\Roman{section}}
\fancyhead[LE,RO]{\thepage}
\fancyhead[LO]{Conclusion générale}
\fancyhead[RE]{Conclusion générale}
\chapter{Conclusion générale}
\minitoc

\addcontentsline{toc}{chapter}{Conclusion générale}

Ce mémoire avait pour objectif principal de proposer une solution informatique permettant d'améliorer la gestion du parc informatique au sein d'une entreprise, à travers une approche méthodique allant de l'analyse des besoins jusqu'à la réalisation d'une application web fonctionnelle. Le travail réalisé s'inscrit dans une démarche professionnelle, en adéquation avec les exigences d'un Master Professionnel et les réalités du monde de l'entreprise.\par

La première partie du mémoire a permis de poser le cadre général du projet, en présentant l'entreprise, son contexte organisationnel et les problématiques liées à la gestion du parc informatique. L'étude de l'existant a mis en évidence plusieurs limites, notamment le manque de centralisation des informations, la difficulté de suivi des équipements et l'absence d'outils permettant une traçabilité fiable des opérations.\par

La phase de conception a constitué une étape clé du projet. À travers l'utilisation de la modélisation UML, les besoins fonctionnels ont été formalisés de manière claire et structurée. Les différents diagrammes ont permis d'identifier les acteurs, les fonctionnalités attendues, les données à manipuler et les règles de gestion à respecter.\par

La réalisation du système a abouti au développement d'une application web de gestion du parc informatique, intégrant les principales fonctionnalités nécessaires : gestion des utilisateurs et des rôles, inventaire des équipements, affectation du matériel, gestion des pannes et des interventions, suivi des opérations de maintenance, mise à la réforme du matériel, ainsi que la sécurisation des accès.\par

Les phases de tests et de validation ont confirmé que l'application répond aux besoins définis lors de la conception. Les fonctionnalités développées sont opérationnelles, les règles de gestion sont respectées et les données sont correctement structurées et sécurisées.\par

Toutefois, ce travail présente également certaines limites, liées notamment au périmètre du projet et au cadre académique dans lequel il a été réalisé. Ces limites ouvrent la voie à de nombreuses perspectives d'évolution, telles que l'enrichissement fonctionnel de l'application, le renforcement de la sécurité, l'intégration de modules de reporting ou encore le déploiement dans un environnement de production à plus grande échelle.\par

En définitive, ce mémoire démontre l'intérêt d'une approche structurée et méthodique dans la conception et la réalisation d'un système d'information. La solution développée constitue une base solide et évolutive, susceptible de répondre durablement aux besoins de gestion du parc informatique et d'accompagner l'entreprise dans l'amélioration de ses processus internes.\par

[1] I. Sommerville , Software Engineering, 10th ed ., Boston, MA, USA: Pearson, 2016.\par

[2] G. Booch , J. Rumbaugh , and I. Jacobson, The Unified Modeling Language User Guide, 2nd ed ., Boston, MA, USA: Addison-Wesley, 2005.\par

[3] J. Rumbaugh , I. Jacobson, and G. Booch , The Unified Modeling Language Reference Manual, 2nd ed ., Boston, MA, USA: Addison-Wesley, 2004.\par

[4] Object Management Group (OMG), Unified Modeling Language (UML) Specification , Version 2.5, 2015. [Online]. Available : https://www.omg.org/spec/UML/\par

[5] A. Dennis, B. H. Wixom , and D. Tegarden , Systems Analysis and Design: An Object- Oriented Approach with UML, 5th ed ., Hoboken, NJ, USA: Wiley , 2015.\par

[6] R. Pressman and B. Maxim, Software Engineering: A Practitioner's Approach , 8th ed ., New York, NY, USA: McGraw-Hill, 2014.\par

[7] PHP Documentation Group, PHP Manual. [Online]. Available : https://www.php.net/docs.php\par

[8] Oracle Corporation, MySQL 8.0 Reference Manual. [Online]. Available : https://dev.mysql.com/doc/\par

[9] M. Fowler, Patterns of Enterprise Application Architecture, Boston, MA, USA: Addison-Wesley, 2003.\par

[10] E. Gamma, R. Helm , R. Johnson, and J. Vlissides , Design Patterns: Elements of Reusable Object- Oriented Software, Boston, MA, USA: Addison-Wesley, 1994.\par

[11] R. Elmasri and S. B. Navathe , Fundamentals of Database Systems , 7th ed ., Boston, MA, USA: Pearson, 2016.\par

[12] K. Laudon and J. Laudon , Management Information Systems : Managing the Digital Firm , 15th ed ., Harlow, U.K.: Pearson, 2018.\par

[13] ISO/IEC 25010, Systems and Software Engineering — Systems and Software Quality Requirements and Evaluation ( SQuaRE ), 2011.\par

[14] OWASP Foundation , OWASP Top 10 – Web Application Security Risks, 2021. [Online]. Available : https://owasp.org/www-project-top-ten/\par

[15] IEEE Computer Society, IEEE Standard for Software and System Test Documentation, IEEE Std 829-2008.\par